\begin{abstract}
Given the increase in the amount of accessible information on the Web, more attention has been drawn to information retrieval systems such as search engines. In web search, recency ranking refers to ranking documents by their relevance to the query, but also taking freshness into account. In this thesis, we propose two models for recency ranking. The first one is the query recency sensitivity model, and the second is a model to predict the publication time of documents. We extract temporal features from several sources and automatically construct ground truth datasets for both models. Furthermore, we integrate the models into an existing commercial search engine with a multi-stage ranking architecture. Our experiments demonstrate an improvement in the effectiveness of the commercial search engine.

\keywords{recency ranking, web search, search engine, machine learning, gradient boosted decision trees, information retrieval, big data}
\end{abstract}

\newpage

% TODO: Navedite naslov na hrvatskom jeziku.
\hrtitle{Modeli rangiranja po svježini za pretraživanje weba}
\begin{sazetak}
Porastom raspoloživih količina dostupnih informacija na Webu, povećalo se zanimanje za sustavima koji dohvaćaju informacije, poput sustava za pretraživanje. Pri pretraživanju weba, rangiranje po svježini odnosi se na rangiranje dokumenata s obzirom na relevantnost na upit, a pritom uključujući i svježinu rezultata. U okviru diplomskog rada prezentiramo dva modela za rangiranje po svježini. Prvi model predviđa osjetljivost upita na svježinu rezultata, a drugi predviđa vrijeme objavljivanja dokumenata. Gradimo vremenske značajke na temelju više izvora i automatski gradimo skup podataka za vrednovanje oba modela. Nadalje, integriramo modele u postojeći komercijalni pretraživač Weba koji se sastoji od višefazne arhitekture za rangiranje. Naši eksperimenti ukazuju na poboljšanje korisnosti komercijalnog pretraživača Weba.

\kljucnerijeci{pretraživanje po svježini, pretraživanje Weba, tražilica, strojno učenje, stabla odluke potpomognuta gradijentom, dohvat informacija, veliki skupovi podataka}
\end{sazetak}
